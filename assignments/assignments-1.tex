\documentclass[11pt]{article}

\input template.tex
\input macros.tex


\title{Python Tutorial}
\date{Spring 2025}
\author{Andreas Mang}

\begin{document}
\maketitle


\section*{Python Namespaces and Variable Scoping}

Test your understanding of Python’s LEGB rule and the use of \texttt{global} and \texttt{nonlocal} keywords.

\subsection*{Tasks}
\noindent What will the following code print? Explain each output:
\begin{lstlisting}[language=Python]
x = "global"

def outer():
    x = "outer"
    def inner():
        print("Inner:", x)
    inner()
    print("Outer:", x)

outer()
print("Global:", x)
\end{lstlisting}



\noindent What will the following code print? Explain the use of \texttt{global}.
\begin{lstlisting}[language=Python]
x = 5

def func():
    global x
    x = 10

func()
print(x)
\end{lstlisting}

\noindent Given the function below, fill in the blank to modify the value of \texttt{x} in the enclosing scope:
\begin{lstlisting}[language=Python]
def outer():
    x = 3
    def inner():
        _____ x
        x = 7
    inner()
    print(x)
\end{lstlisting}

\noindent Explain the LEGB rule using a small custom example with nested functions.


\section*{Basic Python Operators}
We practice using basic Python operators including arithmetic, comparison, logical, assignment, and bitwise operators.

\subsection*{Tasks}
For each question below, write a short Python program or expression that evaluates the result. Show your output and explain briefly how the result was computed.
\begin{enumerate}[label=\arabic*.]
  \item Given \lstinline{a = 12} and \lstinline{b = 5}, compute and print the result of each of the following:
  \begin{enumerate}[label=(\alph*)]
    \item \lstinline{a + b}
    \item \lstinline{a - b}
    \item \lstinline{a * b}
    \item \lstinline{a / b}
    \item \lstinline{a % b}
    \item \lstinline{a // b}
    \item \lstinline{a ** b}
  \end{enumerate}

  \item Evaluate and explain the result of the following comparison expressions:
  \begin{enumerate}[label=(\alph*)]
    \item \lstinline{a == b}
    \item \lstinline{a != b}
    \item \lstinline{a > b}
    \item \lstinline{a <= b}
  \end{enumerate}

  \item Consider \lstinline{x = True} and \lstinline{y = False}. Evaluate:
  \begin{enumerate}[label=(\alph*)]
    \item \lstinline{x and y}
    \item \lstinline{x or y}
    \item \lstinline{not x}
  \end{enumerate}

  \item Starting with \lstinline{c = 7}, apply each of the following assignment operations and print the updated value:
  \begin{enumerate}[label=(\alph*)]
    \item \lstinline{c += 2}
    \item \lstinline{c *= 3}
    \item \lstinline{c -= 4}
    \item \lstinline{c /= 5}
  \end{enumerate}

  \item Given \lstinline{m = 6} and \lstinline{n = 3}, evaluate the following bitwise operations:
  \begin{enumerate}[label=(\alph*)]
    \item \lstinline{m & n}
    \item \lstinline{m | n}
    \item \lstinline{m ^ n}
    \item \lstinline{~m}
    \item \lstinline{m << 1}
    \item \lstinline{m >> 1}
  \end{enumerate}
\end{enumerate}



\section*{Lambda Function and Sorting}

You are given a list of tuples, where each tuple contains a name and a score:
\begin{lstlisting}[language=python]
data = [("Alice", 82), ("Bob", 91), ("Charlie", 78), ("David", 85)]
\end{lstlisting}
\begin{enumerate}
\item Use a \texttt{lambda} function and the \texttt{sorted()} function to sort this list in:
\begin{itemize}
    \item[(a)] Ascending order by score
    \item[(b)] Descending order by name (lexicographically)
\end{itemize}
\item Write the corresponding Python code to display the sorted results.
\end{enumerate}




\section*{Numerical Integration with SciPy}

Use Python and the \texttt{scipy.integrate} module to numerically evaluate the following definite integral:
\[
\int_0^2 \exp(-x^2) \, \text{d}x
\]

\subsection*{Tasks:}
\begin{enumerate}
\item Import the necessary function from SciPy: \lstinline[language=python]{from scipy.integrate import quad}
\item Define the function \( f(x) = \exp(-x^2) \).
\item Use \texttt{quad} to compute the integral over the interval \( [0, 2] \).
\item Print the result and the estimated error.
\end{enumerate}

\noindent\textbf{Bonus:} Try changing the integration bounds to \( [-\infty, \infty] \) and observe the result. What is the significance of this value?





\section*{Robust Calculator with Exception Handling}

Write a Python program that performs division between two numbers entered by the user.
\subsection*{Tasks:}
\begin{enumerate}
\item Prompt the user to enter two numbers (numerator and denominator).
\item Attempt to convert the inputs to \texttt{float}.
\item Perform the division: \texttt{numerator / denominator}.
\item Handle the following exceptions:
\begin{itemize}
    \item \texttt{ValueError}: if the input is not a number.
    \item \texttt{ZeroDivisionError}: if the denominator is zero.
\end{itemize}
\item If the division is successful, print the result.
\item Always print \texttt{"Operation complete."} at the end, regardless of whether an exception occurred.
\end{enumerate}

\subsection*{Sample Output:}
\begin{lstlisting}[language=python]
Enter numerator: 10
Enter denominator: 0
Error: Cannot divide by zero.
Operation complete.
\end{lstlisting}



\section*{Solving a Linear System with an Ill-Conditioned Matrix}

Consider the following linear system \( Ax = b \), where:
\[
A = \begin{bmatrix}
1 & 1 \\
1 & 1.0001
\end{bmatrix},
\quad
b = \begin{bmatrix}
2 \\
2.0001
\end{bmatrix}
\]

\subsection*{Tasks:}
\begin{enumerate}
\item Use NumPy to define the matrix \( A \) and the vector \( b \).
\item Solve for \( x \) using \lstinline[language=python]{numpy.linalg.solve}.
\item Compute the condition number of \( A \) using \lstinline[language=python]{numpy.linalg.cond}.
\item Comment on the stability of the solution and the effect of the condition number.
\end{enumerate}

\subsection*{Bonus:}
Try modifying the second element of \( b \) slightly and observe how the solution changes.




\section*{Object-Oriented Programming}
Understand how to define and use classes in Python by creating objects, methods, and constructors.

\subsection*{Tasks:}
Complete the tasks below by writing Python code and explaining what it does.
\begin{enumerate}[label=\arabic*.]
\item Define a class called \texttt{Circle} that has the following:
\begin{itemize}
    \item An \lstinline{__init__} method that takes a radius
    \item A method \texttt{area()} that returns the area of the circle
    \item A method \texttt{circumference()} that returns the circumference
\end{itemize}
\item Create an instance of \texttt{Circle} with radius 5 and print its area and circumference.
\item Modify the class to include a class variable \lstinline{pi = 3.14159} and use it in the calculations.
\item Add a string representation method \lstinline{__str__} so that printing the object returns a string like \lstinline[language=python]{"Circle with radius 5"}.
\item Create a subclass \texttt{Sphere} that inherits from \texttt{Circle} and adds a method \texttt{volume()} that returns the volume of the sphere.
\item Create an instance of \texttt{Sphere} with radius 5 and print its area (surface area), circumference, and volume.
\end{enumerate}




\end{document}
