\documentclass[mathserif,20pt,xcolor=table,compress,aspectratio=169]{beamer}

\def\titleskip{2.8cm}
\usepackage{multirow}
\usepackage{amsmath,amssymb}
\usepackage[letterpaper,top=0.5in,
bottom=1.5in,left=0.8in,right=1.2in]{geometry}
\usepackage[table]{xcolor}
\usepackage[us,mmddyyyy]{datetime}
\settimeformat{ampmtime}

%\usepackage{fancyhdr}
%\pagestyle{fancy}

\usepackage{comment}
\usepackage{booktabs}
\usepackage{enumitem}
\usepackage{lastpage}
\usepackage{nicefrac}
\usepackage{wasysym}
\usepackage{marvosym}
\usepackage{sansmath}
\sansmath

\usepackage{titling}

\usepackage{setspace}
\renewcommand{\baselinestretch}{1.2}

%\setlength{\droptitle}{-1cm}
%\date{\vspace{-2cm}}

\definecolor{red}{RGB}{200, 16, 46}

\renewcommand*\familydefault{\sfdefault}
\usepackage[colorlinks=true,
filecolor=black,
citecolor=red,
urlcolor=red,
backref=page]{hyperref}
\hypersetup{pdfauthor={Andreas Mang},
linkcolor=red,
,pdflang=en-US,
pdftitle=Homework Assignment,
citecolor=red}

\usepackage[tagged, highstructure]{accessibility}
\usepackage{graphicx}
\usepackage{cleveref}

\newif\ifsolution



\newcolumntype{R}{>{\columncolor{gray!20}}r}
\newcolumntype{L}{>{\columncolor{gray!20}}l}
\newcolumntype{C}{>{\columncolor{gray!20}}c}

\usepackage{listings}

\definecolor{mygreen}{rgb}{0,0.6,0}
\definecolor{mygray}{rgb}{0.5,0.5,0.5}
\definecolor{mymauve}{rgb}{0.58,0,0.82}

\lstset{ %
backgroundcolor=\color{white},      % choose the background color
basicstyle=\ttfamily\normalsize,  % size of fonts used for the code
breaklines=true,                    % automatic line breaking only at whitespace
captionpos=b,                       % sets the caption-position to bottom
commentstyle=\color{mygreen},       % comment style
escapeinside={\%*}{*)},             % if you want to add LaTeX within your code
keywordstyle=\color{blue},          % keyword style
stringstyle=\color{mymauve},        % string literal style
}



\title[]{\large K-Means Clustering}
%\subtitle[]{}
\author[]{Andreas Mang}
\institute[]{Department of Mathematics, Scientific Computing, Optimization, and Parallel Algorithms Lab, University of Houston}
\date[]{Python Workshop, Department of Mathematics, University of Houston}

\begin{document}

\begin{frame}[plain,label=mytitlepage]
\titlepage
\end{frame}


\begin{frame}{What is K-Means Clustering?}
\begin{itemize}
    \item An unsupervised learning algorithm.
    \item Partitions data into $k$ clusters.
    \item Each point belongs to the cluster with the nearest centroid.
    \item Objective: minimize intra-cluster variance.
\end{itemize}
\end{frame}

\begin{frame}{Algorithm Overview}
\begin{enumerate}
    \item Initialize $k$ centroids randomly.
    \item Assign each point to the nearest centroid.
    \item Update centroids as the mean of assigned points.
    \item Repeat steps 2–3 until convergence.
\end{enumerate}
\end{frame}

\begin{frame}[fragile]{Python Libraries}
\begin{lstlisting}[language=Python]
import numpy as np
import matplotlib.pyplot as plt
from sklearn.cluster import KMeans
from sklearn.datasets import make_blobs
\end{lstlisting}
\end{frame}

\begin{frame}[fragile]{Generating Synthetic Data}
\begin{lstlisting}[language=Python]
X, y_true = make_blobs(n_samples=300,
                       centers=4,
                       cluster_std=0.60,
                       random_state=0)

plt.scatter(X[:, 0], X[:, 1], s=50)
plt.title("Generated Data")
plt.show()
\end{lstlisting}
\end{frame}

\begin{frame}[fragile]{Applying K-Means}
\begin{lstlisting}[language=Python]
kmeans = KMeans(n_clusters=4, random_state=0)
kmeans.fit(X)

y_kmeans = kmeans.predict(X)
centroids = kmeans.cluster_centers_
\end{lstlisting}
\end{frame}

\begin{frame}[fragile]{Visualizing Clusters}
\begin{lstlisting}[language=Python]
plt.scatter(X[:, 0], X[:, 1], c=y_kmeans, cmap='viridis')
plt.scatter(centroids[:, 0], centroids[:, 1],
            c='red', s=200, marker='X')
plt.title("K-Means Clustering Results")
plt.show()
\end{lstlisting}
\end{frame}

\begin{frame}[fragile]{Choosing $k$: The Elbow Method}
\begin{lstlisting}[language=Python]
inertia = []
for k in range(1, 10):
    km = KMeans(n_clusters=k, random_state=0)
    km.fit(X)
    inertia.append(km.inertia_)

plt.plot(range(1, 10), inertia, marker='o')
plt.title("Elbow Method")
plt.xlabel("k")
plt.ylabel("Inertia")
plt.show()
\end{lstlisting}
\end{frame}


\begin{frame}{What is the Elbow Method?}
\begin{itemize}
    \item Determine optimal number of clusters $k$.
    \item Compute K-Means for a range of $k$ values.
    \item Plot the \textbf{inertia} (sum of squared distances to centroids).
    \item Look for an "elbow" in the plot where inertia drops off more slowly.
    \item The $k$ at the elbow point is considered a good choice.
\end{itemize}
\end{frame}

\begin{frame}{Key Concepts}
\begin{itemize}
    \item \textbf{Centroids}: Mean of data points in a cluster.
    \item \textbf{Inertia}: Sum of squared distances from each point to its centroid.
    \item \textbf{Convergence}: When centroids no longer move.
\end{itemize}
\end{frame}

\begin{frame}{Limitations of K-Means}
\begin{itemize}
    \item Requires specifying $k$ in advance.
    \item Sensitive to initial centroid placement.
    \item Assumes spherical clusters of similar size.
\end{itemize}
\end{frame}

\begin{frame}{Conclusion}
\begin{itemize}
    \item K-Means is simple and efficient for clustering.
    \item Best used when clusters are well-separated.
    \item Combine with techniques like the Elbow Method for choosing $k$.
\end{itemize}
\end{frame}

\end{document}

