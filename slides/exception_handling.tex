\section{Exception Handling}
\begin{frame}[plain]
\sectionpage
\end{frame}

\begin{frame}{What is Exception Handling?}
\begin{itemize}
    \item Exception handling is a mechanism to handle runtime errors, maintaining the normal flow of program execution.
    \item In Python, errors detected during execution are called \textit{exceptions}.
    \item If an exception occurs, Python will stop executing the code and look for a handler.
\end{itemize}
\end{frame}

\begin{frame}[allowframebreaks]{Basic Syntax}
\begin{itemize}
    \item \texttt{try}: Specifies the block of code to test for errors.
    \item \texttt{except}: Specifies the block of code to execute if an error occurs.
    \item \texttt{else}: Specifies the block of code to execute if no error occurs.
    \item \texttt{finally}: A block of code that will always execute, regardless of whether an exception occurred.
\end{itemize}
\end{frame}


\begin{frame}[fragile]{Example of Try-Except Block}
\begin{lstlisting}[language=Python]
try:
    x = 1 / 0
except ZeroDivisionError:
    print("Cannot divide by zero!")
\end{lstlisting}
\begin{itemize}
\item Here, \texttt{ZeroDivisionError} is raised and handled in the \texttt{except} block.
\end{itemize}
\end{frame}


\begin{frame}{Handling Multiple Exceptions}
\begin{itemize}
\item You can handle different exceptions separately.
\item Example:
\begin{itemize}
    \item \texttt{ValueError}: Raised when an invalid value is passed (e.g., converting a string to an integer).
    \item \texttt{IndexError}: Raised when trying to access an invalid index in a list.
\end{itemize}
\end{itemize}
\end{frame}

\begin{frame}[fragile]{Handling Multiple Exceptions}
\begin{lstlisting}[language=Python]
try:
    num = int(input("Enter a number: "))
    result = 10 / num
except ValueError:
    print("Please enter a valid number.")
except ZeroDivisionError:
    print("Cannot divide by zero!")
\end{lstlisting}
\end{frame}

\begin{frame}{Using Else and Finally}
    \begin{itemize}
        \item The \texttt{else} block will execute if no exception occurs.
        \item The \texttt{finally} block will always execute, regardless of exceptions.
    \end{itemize}
\end{frame}

\begin{frame}[fragile]{Example with Else and Finally}
\begin{lstlisting}[language=Python]
try:
    number = int(input("Enter a number: "))
    result = 10 / number
except ZeroDivisionError:
    print("Error: Division by zero.")
else:
    print("The result is:", result)
finally:
    print("This will always execute.")
\end{lstlisting}
\end{frame}

\begin{frame}{Best Practices}
    \begin{itemize}
        \item Only catch exceptions that you can handle effectively.
        \item Avoid using \texttt{except:} as it can catch unintended exceptions.
        \item Always clean up resources in the \texttt{finally} block (e.g., closing files or network connections).
    \end{itemize}
\end{frame}

\begin{frame}{Conclusion}
    \begin{itemize}
        \item Exception handling is essential for writing robust and fault-tolerant Python programs.
        \item Use \texttt{try}, \texttt{except}, \texttt{else}, and \texttt{finally} effectively to handle errors.
        \item Always ensure your program can recover gracefully from errors or failures.
    \end{itemize}
\end{frame}


