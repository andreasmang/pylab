\begin{frame}{Introduction}
Python is a high-level, interpreted programming language known for its simplicity, readability, and broad applicability in areas ranging from web development and scripting to data science and machine learning. It was created by Guido van Rossum and first released in 1991.
\end{frame}


\begin{frame}[allowframebreaks]{Key Features}
\begin{itemize}
\item \textbf{Easy to Learn and Use}: Python has a clean syntax that mimics natural language.
\item \textbf{Interpreted Language}: Code is executed line-by-line, which aids debugging.
\item \textbf{Dynamic Typing}: Variables do not require an explicit declaration to reserve memory space.
\item \textbf{Extensive Libraries}: Rich standard libraries and active third-party ecosystem.
\item \textbf{Cross-Platform}: Python runs on various platforms including Windows, macOS, and Linux.
\end{itemize}
\end{frame}



\begin{frame}{Resources}
\begin{itemize}
\item
Downloads:
\url{http://www.python.org}
\item
Documentation:
\url{http://www.python.org}
\item Free book:
\url{http://www.python.org}
\end{itemize}
\end{frame}



\section{Basic Syntax}
\begin{frame}[plain]
\sectionpage
\end{frame}


\begin{frame}[fragile]
\frametitle{Basic Syntax}
Python uses indentation to define code blocks instead of braces or keywords.

\begin{lstlisting}[language=python]
print("Hello, world!")
\end{lstlisting}
\end{frame}

\begin{frame}[fragile]
\frametitle{Data Types}
Python supports several built-in data types including:
\begin{itemize}
    \item Numbers: \texttt{int}, \texttt{float}, \texttt{complex}
    \item Text: \texttt{str}
    \item Boolean: \texttt{bool}
    \item Collections: \texttt{list}, \texttt{tuple}, \texttt{set}, \texttt{dict}
\end{itemize}
\end{frame}

\begin{frame}[fragile]
\frametitle{Data Types}
\begin{lstlisting}[language=Python]
x = 5              # int
y = 3.14           # float
name = "Alice"     # str
is_valid = True    # bool
items = [1, 2, 3]  # list
\end{lstlisting}
\end{frame}


\begin{frame}[fragile]
\frametitle{Control Structures}
Python supports conditionals and loops:

\begin{lstlisting}[language=python]
if x > 0:
    print("Positive")

for i in range(5):
    print(i)
\end{lstlisting}
\end{frame}



\begin{frame}[fragile]
\frametitle{Functions}
Functions are defined using the \texttt{def} keyword.
\begin{lstlisting}[language=Python]
def greet(name):
    return f"Hello, {name}!"
\end{lstlisting}
\end{frame}


\begin{frame}{Applications}
Python is widely used in:
\begin{itemize}
\item Web Development (e.g., Django, Flask)
\item Data Science and Machine Learning (e.g., NumPy, pandas, scikit-learn)
\item Automation and Scripting
\item Scientific Computing
\item Game Development
\end{itemize}
\end{frame}


\begin{frame}{Popular Libraries}
\begin{itemize}
    \item \texttt{numpy} – numerical computing
    \item \texttt{pandas} – data analysis
    \item \texttt{matplotlib}, \texttt{seaborn} – plotting
    \item \texttt{scikit-learn} – machine learning
    \item \texttt{flask}, \texttt{django} – web development
\end{itemize}
\end{frame}


\begin{frame}{Next Steps}
\begin{itemize}
    \item Install Python from \texttt{python.org} or use Anaconda
    \item Try interactive coding with Jupyter Notebooks or an online IDE
    \item Practice with basic problems on platforms like LeetCode, HackerRank
\end{itemize}
\end{frame}

\begin{frame}{Conclusion}
Python's versatility and ease of use make it a popular choice among beginners and professionals alike. Its growing ecosystem ensures support for a wide range of programming needs.
\end{frame}
