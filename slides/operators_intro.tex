\section{Basic Python Operators}
\begin{frame}[plain]
\sectionpage
\end{frame}

\begin{frame}{What are Operators?}
\begin{itemize}
    \item Operators are symbols used to perform operations on variables and values.
    \item Operators are classified into several types:
    \begin{itemize}
        \item Arithmetic operators
        \item Comparison operators
        \item Logical operators
        \item Assignment operators
        \item Bitwise operators
    \end{itemize}
\end{itemize}
\end{frame}

\begin{frame}{Arithmetic Operators}
\begin{itemize}
    \item \(\texttt{+}\) Addition
    \item \(\texttt{-}\) Subtraction
    \item \(\texttt{*}\) Multiplication
    \item \(\texttt{/}\) Division (float)
    \item \(\texttt{//}\) Floor division (integer result)
    \item \(\texttt{\%}\) Modulus (remainder)
    \item \(\texttt{**}\) Exponentiation
\end{itemize}
\end{frame}

\begin{frame}[fragile]{Examples}
\begin{lstlisting}[language=python]
a = 10
b = 3

print(a + b)  # Addition
print(a - b)  # Subtraction
print(a * b)  # Multiplication
print(a / b)  # Division
print(a // b) # Floor Division
print(a % b)  # Modulus
print(a ** b) # Exponentiation
\end{lstlisting}
\end{frame}

\begin{frame}{Comparison Operators}
\begin{itemize}
    \item \(\texttt{==}\) Equal to
    \item \(\texttt{!=}\) Not equal to
    \item \(\texttt{<}\) Less than
    \item \(\texttt{>}\) Greater than
    \item \(\texttt{<=}\) Less than or equal to
    \item \(\texttt{>=}\) Greater than or equal to
\end{itemize}
\end{frame}

\begin{frame}[fragile]{Examples}
\begin{lstlisting}[language=python]
a = 10
b = 3

print(a == b)  # Equal to
print(a != b)  # Not equal to
print(a < b)   # Less than
print(a > b)   # Greater than
print(a <= b)  # Less than or equal to
print(a >= b)  # Greater than or equal to
\end{lstlisting}
\end{frame}


\begin{frame}{Logical Operators}
Logical operators are used to combine conditional statements.
\begin{itemize}
    \item \(\texttt{and}\): Returns True if both statements are true
    \item \(\texttt{or}\): Returns True if one of the statements is true
    \item \(\texttt{not}\): Reverses the result, returns True if the result is false
\end{itemize}
\end{frame}

\begin{frame}[fragile]{Examples}
\begin{lstlisting}[language=python]
x = True
y = False

print(x and y)  # False
print(x or y)   # True
print(not x)    # False
\end{lstlisting}
\end{frame}

\begin{frame}{Assignment Operators}
\begin{itemize}
    \item \(\texttt{=}\): Assign
    \item \(\texttt{+=}\): Add and assign
    \item \(\texttt{-=}\): Subtract and assign
    \item \(\texttt{*=}\): Multiply and assign
    \item \(\texttt{/=}\): Divide and assign
    \item \(\texttt{\%=}\): Modulus and assign
    \item \(\texttt{**=}\): Exponentiate and assign
\end{itemize}
\end{frame}

\begin{frame}[fragile]{Examples}
\begin{lstlisting}[language=python]
x = 10
x += 5  # x = x + 5
x -= 3  # x = x - 3
x *= 2  # x = x * 2
x /= 4  # x = x / 4
x %= 3  # x = x % 3
x **= 2 # x = x ** 2
print(x)
\end{lstlisting}
\end{frame}


\begin{frame}{Bitwise Operators}
Bitwise operators are used to perform operations on binary numbers.
\begin{itemize}
    \item \(\texttt{\&}\): Bitwise AND
    \item \(\texttt{|}\): Bitwise OR
    \item \texttt{\textasciicircum}: Bitwise XOR
    \item \texttt{\textasciitilde}: Bitwise NOT
    \item \(\texttt{<<}\): Left shift
    \item \(\texttt{>>}\): Right shift
\end{itemize}
\end{frame}

\begin{frame}[fragile]{Examples}
\begin{lstlisting}[language=python]
a = 10  # 1010 in binary
b = 4   # 0100 in binary

print(a & b)  # Bitwise AND
print(a | b)  # Bitwise OR
print(a ^ b)  # Bitwise XOR
print(~a)     # Bitwise NOT
print(a << 1) # Left shift
print(a >> 1) # Right shift
\end{lstlisting}
\end{frame}

\begin{frame}{Conclusion}
\begin{itemize}
    \item Operators are essential building blocks in Python.
    \item Understanding arithmetic, comparison, logical, assignment, and bitwise operators allows you to perform various operations in your programs.
    \item Practice using these operators will help you write efficient and expressive code.
\end{itemize}
\end{frame}

