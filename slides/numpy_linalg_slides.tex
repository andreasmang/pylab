\section{Numpy: Linear Algebra}
\begin{frame}[plain]
\sectionpage
\end{frame}


\begin{frame}{Overview}
We illustrate basic linear algebra operations using NumPy, a powerful numerical computing library in Python.
\end{frame}



\begin{frame}[fragile]
\frametitle{Matrix Multiplication}
We perform matrix multiplication using the \texttt{np.dot} function:

\begin{lstlisting}[language=Python]
A = np.array([[1, 2], [3, 4]])
B = np.array([[2, 0], [1, 2]])
C = np.dot(A, B)
\end{lstlisting}
\end{frame}


\begin{frame}[fragile]
\frametitle{Determinant of a Matrix}
The determinant of matrix \( A \) is computed using:

\begin{lstlisting}[language=Python]
det_A = np.linalg.det(A)
\end{lstlisting}
\end{frame}


\begin{frame}[fragile]
\frametitle{Inverse of a Matrix}
To find the inverse of matrix \( A \):

\begin{lstlisting}[language=Python]
inv_A = np.linalg.inv(A)
\end{lstlisting}

\end{frame}

\begin{frame}[fragile]
\frametitle{Solving Linear Equations}
To solve the system \( A x = b \):

\begin{lstlisting}[language=Python]
b = np.array([5, 11])
x = np.linalg.solve(A, b)
\end{lstlisting}
\end{frame}


\begin{frame}[fragile]
\frametitle{Eigenvalues and Eigenvectors}
Eigenvalues and eigenvectors of matrix \( A \) are computed as:

\begin{lstlisting}[language=Python]
eigvals, eigvecs = np.linalg.eig(A)
\end{lstlisting}
\end{frame}


\begin{frame}{Conclusion}
NumPy provides a simple and efficient interface for performing linear algebra operations including matrix multiplication, inversion, solving systems of equations, and computing eigenvalues and eigenvectors.
\end{frame}

