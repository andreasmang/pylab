\section{Python Namespaces and Variable Scoping}
\begin{frame}[plain]
\sectionpage
\end{frame}


\begin{frame}{What is a Namespace?}
A \textbf{namespace} is a container that maps names to objects.
\begin{itemize}
    \item \textbf{Local}: Inside a function
    \item \textbf{Enclosing}: In outer functions
    \item \textbf{Global}: Top-level of a script or module
    \item \textbf{Built-in}: Predefined names in Python
\end{itemize}
\end{frame}

\begin{frame}{Variable Scoping}
Python uses the \textbf{LEGB Rule} to resolve names:
\begin{enumerate}
    \item \textbf{L}ocal
    \item \textbf{E}nclosing
    \item \textbf{G}lobal
    \item \textbf{B}uilt-in
\end{enumerate}

Each level is searched in order for a variable name.
\end{frame}


\begin{frame}[fragile]{Global vs Local Scope}
\begin{lstlisting}[language=Python]
x = 10

def func():
    x = 5
    print("Local x:", x)

func()
print("Global x:", x)
\end{lstlisting}
\end{frame}

\begin{frame}[fragile]{Using the \texttt{global} Keyword}
\begin{lstlisting}[language=Python]
x = 10

def modify():
    global x
    x = 20

modify()
print(x)
\end{lstlisting}
\end{frame}

\begin{frame}[fragile]{The \texttt{nonlocal} Keyword}
\begin{lstlisting}[language=Python]
def outer():
    x = "outer"
    def inner():
        nonlocal x
        x = "inner"
    inner()
    print(x)

outer()
\end{lstlisting}
\end{frame}

\begin{frame}[fragile]{Built-in Scope}
Python comes with a set of built-in names.

\begin{lstlisting}[language=Python]
print(len([1, 2, 3]))  # built-in len
len = "Length"
print(len)             # overrides built-in
del len
print(len([1, 2, 3]))  # restores built-in
\end{lstlisting}
\end{frame}

\begin{frame}{Best Practices}
\begin{itemize}
    \item Avoid using \texttt{global} if possible
    \item Use \texttt{nonlocal} with care in nested functions
    \item Don’t shadow built-in names like \texttt{len}, \texttt{sum}, etc.
    \item Keep functions pure and return values instead of modifying outer scopes
\end{itemize}
\end{frame}

