\documentclass[mathserif,20pt,xcolor=table,compress,aspectratio=169]{beamer}

\def\titleskip{2.8cm}
\usepackage{multirow}
\usepackage{amsmath,amssymb}
\usepackage[letterpaper,top=0.5in,
bottom=1.5in,left=0.8in,right=1.2in]{geometry}
\usepackage[table]{xcolor}
\usepackage[us,mmddyyyy]{datetime}
\settimeformat{ampmtime}

%\usepackage{fancyhdr}
%\pagestyle{fancy}

\usepackage{comment}
\usepackage{booktabs}
\usepackage{enumitem}
\usepackage{lastpage}
\usepackage{nicefrac}
\usepackage{wasysym}
\usepackage{marvosym}
\usepackage{sansmath}
\sansmath

\usepackage{titling}

\usepackage{setspace}
\renewcommand{\baselinestretch}{1.2}

%\setlength{\droptitle}{-1cm}
%\date{\vspace{-2cm}}

\definecolor{red}{RGB}{200, 16, 46}

\renewcommand*\familydefault{\sfdefault}
\usepackage[colorlinks=true,
filecolor=black,
citecolor=red,
urlcolor=red,
backref=page]{hyperref}
\hypersetup{pdfauthor={Andreas Mang},
linkcolor=red,
,pdflang=en-US,
pdftitle=Homework Assignment,
citecolor=red}

\usepackage[tagged, highstructure]{accessibility}
\usepackage{graphicx}
\usepackage{cleveref}

\newif\ifsolution



\newcolumntype{R}{>{\columncolor{gray!20}}r}
\newcolumntype{L}{>{\columncolor{gray!20}}l}
\newcolumntype{C}{>{\columncolor{gray!20}}c}

\usepackage{listings}

\definecolor{mygreen}{rgb}{0,0.6,0}
\definecolor{mygray}{rgb}{0.5,0.5,0.5}
\definecolor{mymauve}{rgb}{0.58,0,0.82}

\lstset{ %
backgroundcolor=\color{white},      % choose the background color
basicstyle=\ttfamily\normalsize,  % size of fonts used for the code
breaklines=true,                    % automatic line breaking only at whitespace
captionpos=b,                       % sets the caption-position to bottom
commentstyle=\color{mygreen},       % comment style
escapeinside={\%*}{*)},             % if you want to add LaTeX within your code
keywordstyle=\color{blue},          % keyword style
stringstyle=\color{mymauve},        % string literal style
}



\title[]{\large Spectral Clustering}
%\subtitle[]{}
\author[]{Andreas Mang}
\institute[]{Department of Mathematics, Scientific Computing, Optimization, and Parallel Algorithms Lab, University of Houston}
\date[]{Python Workshop, Department of Mathematics, University of Houston}



\begin{document}

\begin{frame}[plain,label=mytitlepage]
\titlepage
\end{frame}



\begin{frame}{What is Spectral Clustering?}
\begin{itemize}
    \item A graph-based clustering algorithm.
    \item Converts data into a similarity graph.
    \item Computes the graph Laplacian and its eigenvectors.
    \item Performs clustering (e.g., with K-Means) on the reduced eigenvector space.
\end{itemize}
\end{frame}

\begin{frame}{Why Use Spectral Clustering?}
\begin{itemize}
    \item Handles non-convex and complex cluster shapes.
    \item Suitable for data not well separated by linear boundaries.
    \item Leverages the global structure of the data via eigen decomposition.
\end{itemize}
\end{frame}

\begin{frame}[fragile]{Python Libraries and Dataset}
\begin{lstlisting}[language=Python]
from sklearn.datasets import make_moons
X, y_true = make_moons(n_samples=300,
                       noise=0.05,
                       random_state=0)

plt.scatter(X[:, 0], X[:, 1], s=50)
plt.title("Two Moons Dataset")
plt.show()
\end{lstlisting}
\end{frame}

\begin{frame}[fragile]{Applying Spectral Clustering}
\begin{lstlisting}[language=Python]
from sklearn.cluster import SpectralClustering

sc = SpectralClustering(n_clusters=2,
                        affinity='nearest_neighbors',
                        assign_labels='kmeans',
                        random_state=0)

labels = sc.fit_predict(X)
\end{lstlisting}
\end{frame}

\begin{frame}[fragile]{Visualizing the Result}
\begin{lstlisting}[language=Python]
plt.scatter(X[:, 0], X[:, 1], c=labels, cmap='viridis')
plt.title("Spectral Clustering Result")
plt.show()
\end{lstlisting}
\end{frame}

\begin{frame}{Affinity Options}
\texttt{affinity=} controls the similarity measure:
\begin{itemize}
    \item \texttt{nearest\_neighbors} — graph-based neighborhood
    \item \texttt{rbf} — radial basis function kernel
    \item \texttt{precomputed} — custom similarity matrix
\end{itemize}
\end{frame}

\begin{frame}{Key Concepts}
\begin{itemize}
    \item \textbf{Similarity Graph}: Encodes how similar points are.
    \item \textbf{Graph Laplacian}: Captures the structure of the graph.
    \item \textbf{Eigenvectors}: Used for dimensionality reduction before clustering.
\end{itemize}
\end{frame}

\begin{frame}{Limitations and Notes}
\begin{itemize}
    \item Computationally expensive for large datasets.
    \item Choice of affinity and number of neighbors impacts performance.
    \item Often requires tuning hyperparameters.
\end{itemize}
\end{frame}

\begin{frame}{Conclusion}
\begin{itemize}
    \item Spectral Clustering excels in identifying non-linear patterns.
    \item Useful for graph-based and structured datasets.
    \item Combine with visualization and elbow-like heuristics for best results.
\end{itemize}
\end{frame}

\end{document}
