\section{Numpy: Introduction}
\begin{frame}[plain]
\sectionpage
\end{frame}


\begin{frame}{Overview}
This document provides an introduction to the NumPy library in Python, covering array creation, attributes, indexing, operations, and useful functions.
\end{frame}


\begin{frame}[fragile]
\frametitle{Creating Arrays}
NumPy arrays can be one-dimensional or multi-dimensional:

\begin{lstlisting}[language=Python]
a = np.array([1, 2, 3])
b = np.array([[1, 2, 3], [4, 5, 6]])
\end{lstlisting}

\end{frame}


\begin{frame}[fragile]
\frametitle{Array Attributes}
You can inspect properties of an array, such as its shape, size, and data type:

\begin{lstlisting}[language=Python]
b.shape  # (2, 3)
b.size   # 6
b.dtype  # int64 (or similar depending on system)
\end{lstlisting}

\end{frame}

\begin{frame}[fragile]
\frametitle{Array Indexing and Slicing}
Access specific elements or slices of the array:

\begin{lstlisting}[language=Python]
b[0]     # First row
b[1, 2]  # Element at second row, third column
\end{lstlisting}

\end{frame}


\begin{frame}[fragile]
\frametitle{Array Operations}
Arrays can be added or multiplied using standard operators. The dot product is also available:

\begin{lstlisting}[language=Python]
x = np.array([1, 2, 3])
y = np.array([4, 5, 6])
x + y          # Element-wise addition
np.dot(x, y)   # Dot product
\end{lstlisting}

\end{frame}


\begin{frame}[fragile]
\frametitle{Useful Functions}
NumPy offers convenient functions to generate arrays:

\begin{lstlisting}[language=Python]
np.zeros((2, 3))     # Array of zeros
np.ones((2, 3))      # Array of ones
np.arange(0, 10, 2)  # Array from 0 to 10 with step 2
np.linspace(0, 1, 5) # 5 equally spaced numbers from 0 to 1
\end{lstlisting}
\end{frame}

\begin{frame}{Conclusion}
NumPy provides efficient array structures and numerical operations essential for scientific computing in Python.
\end{frame}

