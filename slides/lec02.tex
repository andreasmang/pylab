\documentclass[mathserif,20pt,xcolor=table,compress,aspectratio=169]{beamer}

\def\titleskip{2.8cm}
\usepackage{multirow}
\usepackage{amsmath,amssymb}
\usepackage[letterpaper,top=0.5in,
bottom=1.5in,left=0.8in,right=1.2in]{geometry}
\usepackage[table]{xcolor}
\usepackage[us,mmddyyyy]{datetime}
\settimeformat{ampmtime}

%\usepackage{fancyhdr}
%\pagestyle{fancy}

\usepackage{comment}
\usepackage{booktabs}
\usepackage{enumitem}
\usepackage{lastpage}
\usepackage{nicefrac}
\usepackage{wasysym}
\usepackage{marvosym}
\usepackage{sansmath}
\sansmath

\usepackage{titling}

\usepackage{setspace}
\renewcommand{\baselinestretch}{1.2}

%\setlength{\droptitle}{-1cm}
%\date{\vspace{-2cm}}

\definecolor{red}{RGB}{200, 16, 46}

\renewcommand*\familydefault{\sfdefault}
\usepackage[colorlinks=true,
filecolor=black,
citecolor=red,
urlcolor=red,
backref=page]{hyperref}
\hypersetup{pdfauthor={Andreas Mang},
linkcolor=red,
,pdflang=en-US,
pdftitle=Homework Assignment,
citecolor=red}

\usepackage[tagged, highstructure]{accessibility}
\usepackage{graphicx}
\usepackage{cleveref}

\newif\ifsolution



\newcolumntype{R}{>{\columncolor{gray!20}}r}
\newcolumntype{L}{>{\columncolor{gray!20}}l}
\newcolumntype{C}{>{\columncolor{gray!20}}c}

\usepackage{listings}

\definecolor{mygreen}{rgb}{0,0.6,0}
\definecolor{mygray}{rgb}{0.5,0.5,0.5}
\definecolor{mymauve}{rgb}{0.58,0,0.82}

\lstset{ %
backgroundcolor=\color{white},      % choose the background color
basicstyle=\ttfamily\normalsize,  % size of fonts used for the code
breaklines=true,                    % automatic line breaking only at whitespace
captionpos=b,                       % sets the caption-position to bottom
commentstyle=\color{mygreen},       % comment style
escapeinside={\%*}{*)},             % if you want to add LaTeX within your code
keywordstyle=\color{blue},          % keyword style
stringstyle=\color{mymauve},        % string literal style
}



\title[]{\large Solving a 2D Elliptic PDE with Finite Differences}
%\subtitle[]{}
\author[]{Andreas Mang}
\institute[]{Department of Mathematics, Scientific Computing, Optimization, and Parallel Algorithms Lab, University of Houston}
\date[]{Python Workshop, Department of Mathematics, University of Houston}

\begin{document}

\begin{frame}[plain,label=mytitlepage]
\titlepage
\end{frame}


\begin{frame}{Problem Statement}
We consider the 2D Poisson equation on the unit square:
\[
- \Delta u(x, y) = f(x, y), \quad (x, y) \in (0, 1)^2
\]
with homogeneous Dirichlet boundary conditions:
\[
u(x, y) = 0, \quad \text{for } (x, y) \in \partial \Omega
\]
\end{frame}


\begin{frame}{Laplace Operator}
The Laplace operator in 2D is given by:
\[
\Delta u = \frac{\partial^2 u}{\partial x^2} + \frac{\partial^2 u}{\partial y^2}
\]
This operator describes diffusion, electrostatics, and other steady-state phenomena.
\end{frame}


\begin{frame}{Source Term}
We choose a smooth, separable function:
\[
f(x, y) = \sin(\pi x)\sin(\pi y)
\]
This function ensures the solution is smooth and bounded.
\end{frame}


\begin{frame}{Discretization}
\begin{itemize}
    \item Domain discretized into a uniform grid of size \( (n_x+2) \times (n_y+2) \)
    \item Grid spacings: \( h_x = \frac{1}{n_x+1}, \quad h_y = \frac{1}{n_y+1} \)
    \item Use second-order central finite differences
\end{itemize}
\[
\frac{u_{i+1,j} - 2u_{i,j} + u_{i-1,j}}{h_x^2} + \frac{u_{i,j+1} - 2u_{i,j} + u_{i,j-1}}{h_y^2}
\]
\end{frame}


\begin{frame}{Linear System}
The discretized PDE results in a linear system:
\[
A u = f
\]
\begin{itemize}
    \item \( A \): Sparse matrix representing 2D Laplacian
    \item \( u \): Vector of unknown values at grid points
    \item \( f \): Vector from evaluating source term
\end{itemize}
\end{frame}

\begin{frame}{Numerical Solution}
\begin{itemize}
    \item Use \texttt{scipy.sparse} for matrix construction
    \item Solve with \texttt{spsolve}
    \item Reshape the solution into 2D for visualization
\end{itemize}
\end{frame}

\begin{frame}{Summary}
\begin{itemize}
    \item Solved 2D Poisson equation using finite differences
    \item Used sparse matrix techniques for efficiency
    \item Visualized the solution to verify behavior
\end{itemize}
\end{frame}

\end{document}

