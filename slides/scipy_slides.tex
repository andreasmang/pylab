\section{Scipy}

\begin{frame}[plain]
\sectionpage
\end{frame}


\begin{frame}{Notebook}

The notebook covers four main functionalities:
\begin{enumerate}
    \item Curve Fitting
    \item Numerical Integration
    \item Solving Linear Equations
    \item Statistical Testing
\end{enumerate}
\end{frame}

\begin{frame}[allowframebreaks]{Content}
\begin{itemize}
\item Special functions (scipy.special)
\item  Integration (scipy.integrate)
\item Optimization (scipy.optimize)
\item Interpolation (scipy.interpolate)
\item Fourier Transforms (scipy.fft)
\item Signal Processing (scipy.signal)
\item Linear Algebra (scipy.linalg)
\item Sparse eigenvalue problems with ARPACK
\item Compressed Sparse Graph Routines (scipy.sparse.csgraph)
\item Spatial data structures and algorithms (scipy.spatial)
\item Statistics (scipy.stats)
\item Multidimensional image processing (scipy.ndimage)
\item File IO (scipy.io)
\end{itemize}
\end{frame}



\begin{frame}[fragile]
\frametitle{Curve Fitting}
We simulate noisy sine wave data and fit it using a sinusoidal model:

\begin{lstlisting}[language=python]
x = np.linspace(0, 10, 100)
y = 3.5 * np.sin(x) + np.random.normal(size=100)
def model_func(x, a, b): return a * np.sin(b * x)
params, _ = optimize.curve_fit(model_func, x, y)
\end{lstlisting}


This returns the best-fit parameters for amplitude and frequency.
\end{frame}


\begin{frame}[fragile]
\frametitle{Integration}
We compute the definite integral of the Gaussian function:

\[
\int_0^{\infty} \exp(-x^2) \,\text{d}x
\]

\begin{lstlisting}[language=python]
result,error = integrate.quad(lambda x: np.exp(-x**2), 0, np.inf)
\end{lstlisting}

\end{frame}



\begin{frame}[fragile]
\frametitle{Linear Algebra}
We solve a system $Ax = y$:
\[
A = \begin{bmatrix} 3 & 1 \\ 1 & 2 \end{bmatrix}, \quad y = \begin{bmatrix} 9 \\ 8 \end{bmatrix}
\]
%
\begin{lstlisting}[language=python]
x_sol = linalg.solve(A, y)
\end{lstlisting}

\end{frame}

\begin{frame}[fragile]
\frametitle{Statistical Testing}
We compare two groups using a two-sample t-test:
\begin{lstlisting}[language=python]
group1 = np.random.normal(5.0, 1.0, 30)
group2 = np.random.normal(5.5, 1.0, 30)
t_stat, p_val = stats.ttest_ind(group1, group2)
\end{lstlisting}

This determines whether the groups have significantly different means.
\end{frame}

\begin{frame}{Conclusion}
SciPy provides powerful tools for scientific computing, all within a consistent and easy-to-use interface.
\end{frame}
